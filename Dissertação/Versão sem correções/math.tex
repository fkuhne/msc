%%%%%%%%%%%%%%%%%%%%%%%%%%%%%%%%%%%%%%%%%
\section{Colchetes de Lie}\label{sec:lie}
Seja $f$ e $g$ dois campos vetoriais independentes. O colchete de Lie de $f$ e $g$, denotado por $[f,g]$, � um terceiro campo vetorial definido como
\[ [f,g](x)=\frac{\partial g}{\partial x}f(x) - \frac{\partial f}{\partial x}g(x) \]
onde $\frac{\partial f}{\partial x}$ e $\frac{\partial g}{\partial x}$ s�o matrizes jacobianas. A seguinte nota��o � utilizada:
\begin{align*}
	ad^0_fg(x) &= g(x) \\
	ad_fg(x) &= [f,g](x) \\
	ad^k_fg(x) &= [f,ad^{k-1}_fg](x)
\end{align*}

Os colchetes de Lie apresentam ainda as seguintes propriedades:
\begin{itemize}
\item Bilinearidade: sejam $f_1$, $f_2$, $g_1$ e $g_2$ campos vetoriais e $\alpha$, $\beta$ n�meros reais. Ent�o,
\begin{align*}
	\left[ \alpha f_1+\beta f_2,g_1\right] &= \alpha\left[f_1,g_1\right]+\beta\left[f_2,g_1\right] \\
	\left[f_1, \alpha g_1+\beta f_2\right] &= \alpha\left[f_1,g_1\right]+\beta\left[f_1,g_2\right]
\end{align*}
\item Comutatividade:
\[ \left[f,g\right] = -\left[g,f\right] \]
\item Identidade de Jacobi: se $f$ e $g$ s�o campos vetoriais e $h$ � uma fun��o real, ent�o
\[ L_{[f,g]}h(x)=L_fL_gh(x)-L_gL_fh(x), \]
\end{itemize}
onde $L_fh(x)$ � a {\em derivada de Lie},
\[ L_fh(x)=\frac{\partial h}{\partial x}f(x) \]
e $L_gL_fh(x) = \frac{\partial \left(L_fh\right)}{\partial x}g(x)$.


%%%%%%%%%%%%%%%%%%%%%%%%%%%%%%%%%%%%%%%%%%%%%%%%%%
\section{Distribui��es Involutivas}\label{sec:inv}
Uma distribui��o $\Delta$ � involutiva se � fechada no colchete de Lie:
\[ g_1\in\Delta \text{~~e~~} g_2\in\Delta\Rightarrow\left[g_1,g_2\right]\in\Delta \]

Se $\Delta$ � uma distribui��o n�o singular (ou seja, $\Delta$ � igual para todo e qualquer $x$), gerada pelos campos vetoriais linearmente independentes $f_1,\ldots,f_r$, ent�o $\Delta$ � involutiva se e somente se
\[ \left[f_i,f_j\right]\in\Delta,\quad\forall~1\leq i,j\leq r, \]
ou
\[ \left[f_i,f_j\right](x) = \sum_{k=1}^r\alpha_{ijk}(x)f_k(x), \quad \forall~i,j, \]
onde $\alpha_{ijk}:\real^n\rightarrow\real$ s�o fun��es escalares e $n$ � a dimens�o de $x$.

Involutividade significa que, se formado o colchete de Lie de qualquer par de campos vetoriais $f_i,f_j$ do conjunto $f_1,\ldots,f_r$, ent�o o campo vetorial resultante pode ser expresso como combina��o linear dos campos vetoriais originais. Nota-se que:
\begin{itemize}
\item Campos vetoriais constantes s�o sempre involutivos. De fato, o colchete de Lie de dois campos vetoriais constantes, $[f_{c1},f_{c2}]$, � simplesmente o vetor zero, que pode ser trivialmente expresso como combina��o linear de $f_{c1}$ e $f_{c2}$;
\item Um conjunto composto de apenas um campo vetorial � involutivo. De fato,
\[ \left[f,f\right]=\frac{\partial f}{\partial x}f-\frac{\partial f}{\partial x}f=0 \]
\item Para uma distribui��o ser involutiva, a seguinte condi��o deve ser atendida:
\[ rank\left[f_1(x)~~\cdots~~f_r(x)\right]=rank\left[f_1(x)~~\cdots~~f_r(x)~~[f_i,f_j](x)\right], \]
para todos $i$, $j$ e $x$.
\end{itemize}

O {\em fechamento involutivo} de uma distribui��o $\Delta$ � denotado por $\inv\Delta$.