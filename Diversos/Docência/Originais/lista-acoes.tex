
%%% Local Variables: 
%%% mode: latex
%%% TeX-master: "lista1"
%%% End: 
\documentclass[11pt,epsf]{report}
\setlength{\vsize}{297mm}
\setlength{\hsize}{210mm}
\setlength{\textheight}{230mm}
\setlength{\textwidth}{170mm}
%\voffset -2.5cm
\hoffset -2.5cm

\usepackage[latin1]{inputenc}
%\usepackage{latexsym}
%\usepackage{amssymb}
\usepackage[dvips]{graphicx}

\newenvironment{alternativas}{\renewcommand{\labelenumi}{(\alph{enumi})}\begin{enumerate}\addtolength{\itemsep}{-2.5mm}\setlength{\parsep}{0pt}}{\end{enumerate}}

\newcounter{qcounter}
\newenvironment{question}{\stepcounter{qcounter}\paragraph{\arabic{qcounter}.}}{}




\begin{document}

\begin{center}
ENG04035 - Sistemas de Controle I \\
Profs. Romeu Reginatto e Trist�o Garcia \\
Princ�pios de Realimenta��o e A��es B�sicas de Controle
\end{center}


\begin{question}
Ler o cap�tulo 4, ``Basic Properties of Feedback'', do livro:
G.F. Franklin, J.D. Powell, A.E. Naeini; {\bf Feedback Control of Dynamic Systems}; 
778 p.; 9 caps.; 6 ap�ndices; 24 cm x 20 cm x 3.5cm; alf.; br.; 3$^a$ Ed; 
Addison-Wesley, 1994.
Para este momento, podem ser desconsideradas as se��es 4.2.5, 4.2.7 e 4.3.3.



 Em seguida, resolver os seguintes problemas propostos no livro 
(in�cio na pag. 224): 4.4, 4.5, 4.8(a,b), 4.9, {\em 4.10},
 4.11, 4.12, 4.14, 4.15(a,b,c,d), 4.19(a,b), 4.25(a,b),
4.32(a,c), 4.36(a). Os problemas complementares 
foram aqui marcados em it�lico.
\end{question}



\end{document}











