\documentclass{article}

\usepackage[portuges,brazil]{babel}
\usepackage{isolatin1}

\usepackage{a4wide}

\usepackage{graphicx,psfrag}
\usepackage{float}
\usepackage{subfigure}
\usepackage[rflt]{floatflt}


\newcounter{qcounter}
\newenvironment{question}{\stepcounter{qcounter}\paragraph{\arabic{qcounter}.}}{}
\newenvironment{alternativas}{\renewcommand{\labelenumi}{(\alph{enumi})}\begin{enumerate}\addtolength{\itemsep}{-2.5mm}\setlength{\parsep}{0pt}}{\end{enumerate}}

\newcommand{\matriz}[2]{\left[\begin{array}{#1} #2 \end{array}\right]}

\markright{ENG04035 - Sistemas de Controle I - DELET/UFRGS}
\pagestyle{myheadings}

\begin{document}

\begin{center}
ENG04035 - Sistemas de Controle I - 2001/2 \\
Lista de Exerc�cios - Profs. Romeu Reginatto e Jo�o Manoel Gomes da Silva \\
Lugar das Ra�zes  \\
\end{center}


\begin{question}
Considere os seguintes processos:
{
\Large
$$
\begin{array}{ll}
G_1(s)=\frac{5}{(s+2)(s+10)} & G_4(s) = \frac{1}{s^2+2s+2} \\
G_2(s)=\frac{1}{s(s+a)^2} & G_5(s)=\frac{(s+5)}{(s-2)(s+10)} \\
G_3(s)=\frac{s-5}{(s+2)(s+10)} & G_6(s)=\frac{s+1}{(s+20)(s^2+25)} 
\end{array}
$$
}

Estude o controle proporcional ($C(s)=k$) e 
o controle integral ($C(s)=k_i/s$) de cada um destes
processos pelo m�todo do lugar das ra�zes, utilizando as diretrizes 
abaixo.
\begin{alternativas}
\item
Para quais valores de $k$ ($k_i$) o sistema em malha fechada � 
BIBO-est�vel?
\item
Qual o valor do ganho cr�tico?
\item
Para quais valores de $k$ ($k_i$) a fun��o de transfer�ncia
de malha fechada possui todos os seus p�los reais?
\item
Dentro das poss�veis ajustes dos p�los do sistema realimentado atrav�s
do ganho $k$ ($k_i$), identifique o ajuste que conduz 
aproximadamente ao menor tempo de acomoda��o poss�vel
na resposta ao degrau. Qual o valor do ganho $k$ ($k_i$)
que ajusta esta condi��o?
\item
Obtenba o valor do ganho $k$ ($k_i$), se existir,
que conduz a $MG=10dB$ (margem de ganho). 
\item
Existe possibilidade de ajuste de p�los para o sistema 
realimentado tais que o m�xima sobre-eleva��o 
(m�ximo {\em overshoot}) � inferior a $20\%$? 
Determina a faixa de ganho aproximada que garante esta
condi��o.
\item 
O aumento do ganho $k$ ($k_i$) aumenta ou diminiu as margens
de estabilidade do sistema realimentado?
\end{alternativas}
\end{question}


\begin{question}
Considere o circuito el�trico da figura~\ref{fig:creal}. Determine
o lugar dos p�los da fun��o de transfer�ncia $T(s)=E_o(s)/E_i(s)$
para a varia��o do resistor $R_1$ de $0$ a $+\infty$.

\begin{figure}[H]
\begin{center}
\includegraphics[width=12cm]{evansfig.jpg}
\end{center}
\caption{Circuito el�trico com amplificadores operacionais {\bf ideais}.}
\label{fig:creal}
\end{figure}

\end{question}



\end{document}

\begin{alternativas}
\item[(a)] 
\item[(b)] 
\item[(c)] 
\item[(d)] 
\item[(e)] 
\end{alternativas}













