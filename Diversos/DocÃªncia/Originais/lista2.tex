%%% \emph{}
%%% Local Variables: 
%%% mode: latex
%%% TeX-master: "lista1"
%%% End: 
\documentclass[11pt,epsf]{report}
\setlength{\vsize}{297mm}
\setlength{\hsize}{210mm}
\setlength{\textheight}{230mm}
\setlength{\textwidth}{170mm}
%\voffset -2.5cm
\hoffset -2.5cm

\usepackage[latin1]{inputenc}
%\usepackage{latexsym}
%\usepackage{amssymb}
\usepackage[dvips]{graphicx}


\renewcommand\contentsname{Conte\'udo}
\renewcommand\listfigurename{Lista de Figuras}
\renewcommand\listtablename{Lista de Tabelas}
\renewcommand\bibname{Bibliografia}
\renewcommand\indexname{Indice}
\renewcommand\figurename{Figura}
\renewcommand\tablename{Tabela}
\renewcommand\partname{Parte}
\renewcommand\chaptername{Cap�tulo}
\renewcommand\appendixname{Ap\^endice}
\renewcommand\abstractname{Resumo}
\renewcommand\today{\ifcase\month\or
  Janeiro\or Fevereiro\or Mar�o\or Abril\or Maio\or Junho\or
  Julho\or Agosto\or Setembro\or Outubro\or Novembro\or Dezembror\fi
  \space\number\day, \number\year}

\newtheorem{algo}{Algoritmo}
\newcommand\balgo{\begin{algo}}
\newcommand\ealgo{\end{algo}}
\newtheorem{theor}{Teorema}
\newcommand\bth{\begin{theor}}
\newcommand\eth{\end{theor}}
\newcommand\fpreuve {$\Box$ \\}
\newtheorem{assum}{Hip\'otese}
\newcommand\bhyp{\begin{hyp}}
\newcommand\ehyp{\end{hyp}}
\newcommand\fhyp {$\lhd$ \\}
\newtheorem{defin}{Defini\c c\~ao}
\newcommand\bdefi{\begin{defin}}
\newcommand\edefi{\end{defin}}
\newtheorem{lemma}{Lema}
\newcommand\blem{\begin{lemma}}
\newcommand\elem{\end{lemma}}
\newtheorem{propo}{Proposi\c{c}\~{a}o}
\newcommand\bpropo{\begin{propo}}
\newcommand\epropo{\end{propo}}
\newtheorem{propr}{Propriedade}
\newcommand\bpropr{\begin{propr}}
\newcommand\epropr{\end{propr}}
\newtheorem{coroll}{Corol\'ario}
\newcommand\bcor{\begin{coroll}}
\newcommand\ecor{\end{coroll}}
\newtheorem{exemp}{Exemplo}
\newcommand\bex{\begin{exemp}}
\newcommand\eex{\end{exemp}}
\newcommand \fexe {$\diamondsuit$ \\}
\newtheorem{remarq}{Observa\c c\~ao}
\newcommand\brem{\begin{remarq}}
\newcommand\erem{\end{remarq}}




\newcommand\ipar{\hspace*{\parindent}}
\newcommand\epar{\vspace{\baselineskip}}
\newcommand\pdef{\stackrel{\bigtriangleup}{=}}
\newcommand\bdm{\begin{displaymath}}
\newcommand\edm{\end{displaymath}}
\newcommand\barray{\begin{array}}
\newcommand\earray{\end{array}}
\newcommand\beqa{\begin{eqnarray}}
\newcommand\eeqa{\end{eqnarray}}
\newcommand\beqan{\begin{eqnarray*}}
\newcommand\eeqan{\end{eqnarray*}}
\newcommand\beq{\begin{equation}}
\newcommand\eeq{\end{equation}}
\newcommand\bc{\begin{center}}
\newcommand\ec{\end{center}}
\newcommand\bit{\begin{itemize}}
\newcommand\eit{\end{itemize}}
\newcommand\bquo{\begin{quote}}
\newcommand\equo{\end{quote}}
\newcommand\ben{\begin{enumerate}}
\newcommand\een{\end{enumerate}}
\newcommand{\ignore}[1]{\iffalse #1 \fi}
\newcommand\tz{{\cal Z}}
\newcommand\tzi{{\cal Z}^{-1}}


\begin{document}

\begin{center}
\Large{{\sf LISTA DE EXERC�CIOS 2}}
\end{center}

\begin{enumerate}

\item Ler o cap�tulo 4 do livro: 

G. Franklin, D. Powell, A. Emami-Naeini. {\em Feedback Control of Dynamic
 Systems }


\item Resolver os seguintes exerc�cios propostos da refer�ncia do 
item anterior: 4.2, 4.5, 4.8, 4.11, 4.14, 4.18, 4.19, 4.25, 4.32,
4.36.

\item Considere o diagrama em blocos da figura 1

\begin{figure}[!hhh]
\vspace*{5cm}
\caption{sistema gen�rico}
\end{figure} 

\begin{enumerate}
\item Supondo que $G_1(s),G_2(s)$ n�o possuam zeros na origem 
($s=0$), qual a condi��o necess�ria e suficiente, em termos de n�mero de 
p�los na origem em $G_1(s)$ e $G_2(s)$  
e das ra�zes da equa��o caracter�stica $1+G_1(s)G_2(s)$,
 para que o sistema rejeite assintoticamente uma perturba��o do tipo salto?  
\item Supondo que $G_2(s)$ tenha um zero na origem, qual a condi��o necess�ria e suficiente, em termos de n�mero de 
p�los na origem em $G_1(s)$ e $G_2(s)$  
das ra�zes da equa��o caracter�stica $1+G_1(s)G_2(s)$,
 para que o sistema rejeite assintoticamente uma perturba��o do tipo salto?

\end{enumerate}

\item Considere o diagrama em blocos da figura 1

\begin{enumerate}
\item Que condi��es devem ser satisfeitas para que seja 
poss�vel seguir uma refer�ncia senoidal, com freq��ncia 
angular $w$rad/s, com erro nulo em regime permanente.

\item Que condi��es devem ser satisfeitas para que seja 
poss�vel rejeitar assintoticamente em regime permanente
uma refer�ncia senoidal, com freq��ncia 
angular $w$rad/s?
\end{enumerate} 


\end{enumerate}

\end{document}



















