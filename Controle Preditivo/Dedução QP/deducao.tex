\documentclass[10pt, a4paper]{article}
\usepackage[latin1]{inputenc}
\usepackage[brazil]{babel}
\usepackage[dvips,ps2pdf]{hyperref}
\usepackage{amsthm, amsfonts, amssymb, amsmath, graphicx}
\pagestyle{plain}

% Define �rea da folha
\setlength{\vsize}{297mm} \setlength{\hsize}{210mm}
\setlength{\textheight}{245mm} \setlength{\textwidth}{165mm}
\voffset -2.4cm \hoffset -2.5cm

\begin{document}

Considerando o sistema:
\begin{equation*}
	x(k+1) = Ax(k)+Bu(k) \\
\end{equation*}

Fun��o de custo:
\begin{equation*}
	\Phi(k) = \sum_{j=1}^Nx^T(k+j|k)Qx(k+j|k) + u^T(k+j-1|k)Ru(k+j-1|k),
\end{equation*}
onde $N$ � o horizonte de predi��o.

Recursivamente aplicando
\begin{equation*}
	x(k+j|k) = A^jx(k|k) + \sum_{i=0}^{j-1}A^{j-1-i}Bu(k+i|k)
\end{equation*}
e escrevendo
\begin{equation}\label{xbar}
	\overline x = \overline A x(k|k) + S\overline u,
\end{equation}
onde
\begin{equation*}
	\overline x \triangleq \begin{bmatrix}
		x(k+1|k) \\ x(k+2|k) \\ \vdots \\ x(k+N|k)
	\end{bmatrix}
\end{equation*}	
\begin{equation*}
	\overline u \triangleq \begin{bmatrix}
		u(k|k) \\ u(k+1|k) \\ \vdots \\ u(k+N-1|k)
	\end{bmatrix}
\end{equation*}
\begin{equation*}
	\overline A \triangleq \begin{bmatrix}
		A \\ A^2 \\ \vdots \\ A^N
	\end{bmatrix}
\end{equation*}	
\begin{equation*}
	{\bf S} \triangleq \begin{bmatrix}
		B                 & {\bf 0} 					& \cdots & {\bf 0}       \\
		AB       & B      			& \cdots & {\bf 0}       \\
		\vdots	                 & \vdots		     		& \ddots & \vdots        \\
		A^{N-1}B & A^{N-2}B & \cdots & B
	\end{bmatrix},
\end{equation*}
tem-se que:
\begin{equation}\label{phi}
	\Phi(k) = \overline x^T\overline Q\overline x + \overline u^T\overline R\overline u,
\end{equation}
onde
\begin{align*}
	\overline Q &\triangleq \begin{bmatrix}
		Q & {\bf 0} & \cdots & {\bf 0} \\
		{\bf 0} & Q & \cdots & {\bf 0} \\
		\vdots  & \vdots  & \ddots & \vdots  \\
		{\bf 0} & {\bf 0} & \cdots & Q \\
	\end{bmatrix} \qquad
	\overline R \triangleq \begin{bmatrix}
		R & {\bf 0} & \cdots & {\bf 0} \\
		{\bf 0} & R & \cdots & {\bf 0} \\
		\vdots  & \vdots  & \ddots & \vdots  \\
		{\bf 0} & {\bf 0} & \cdots & R \\
	\end{bmatrix}
\end{align*} 

Expandindo a fun��o de custo (\ref{phi}) com (\ref{xbar}), tem-se:
\begin{align*}
	\Phi(k) &= \left[x^T(k|k)\overline A^T + \overline u^TS^T\right]\overline Q\left[\overline A^Tx(k|k)+S\overline u\right] + \overline u^T\overline R\overline u \\
	     &= x^T(k|k)\overline A^T\overline Q\overline Ax(k|k) + x^T(k|k)\overline A^T\overline QS\overline u + \overline u^TS^T\overline Q\overline Ax(k|k) + \overline u^TS^T\overline QS\overline u + \overline u^T\overline R\overline u
\end{align*}

Os termos quadr�ticos em $\overline u$ s�o
\begin{align*}
	\frac{1}{2}\overline u^TH\overline u &= \overline u^TS^T\overline QS\overline u + \overline u^T\overline R\overline u \\
								  &= \overline u^T\left(S^T\overline QS+\overline R\right)\overline u
\end{align*}	
e
\begin{equation*}
	H = 2\left(S^T\overline QS+\overline R\right)
\end{equation*}

Os termos lineares em $\overline u$ s�o:
\begin{align*}
	f^T\overline u &= x^T(k|k)\overline A^T\overline QS\overline u + \overline u^TS^T\overline Q\overline Ax(k|k) \\
				&= 2x^T(k|k)\overline A^T\overline QS\overline u
\end{align*}
e
\begin{equation*}
	f = 2S^T\overline Q\overline Ax(k|k)
\end{equation*}

Assim, $\Phi(k)$ � escrita na forma quadr�tica padr�o
\begin{equation*}
	\Phi(k) = \frac{1}{2}\overline u^T\overline H\overline u + f^T\overline u + c
\end{equation*}
onde $c$ � independente de $\overline u$ e n�o importa para o problema de minimiza��o.

\end{document}